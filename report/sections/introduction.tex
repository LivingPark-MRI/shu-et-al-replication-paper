Parkinson's disease (PD) involves the progressive degeneration of brain neurons, significantly reducing dopamine and affecting motor control~\cite{fahn_2003}. 
This leads to symptoms like tremors and stiffness, and extends beyond motor functions, underscoring the disorder's multifaceted nature~\cite{xia_yao_ye_cheng_2020}. 
Given the variability in how PD manifests and progresses among individuals, accurately assessing the disease's trajectory and rate of progression, especially in its 
early stages, presents significant challenges~\cite{saeed_compagnone_aviv_strafella_black_lang_masellis_2017}~\cite{seyed-mohammad_fereshtehnejad_yashar_zeighami_dagher_postuma_2017}. 
In response to these challenges, certain studies have employed machine learning (ML) techniques to analyze various types of data beyond traditional demographic and clinical data. 
For instance, some of these studies have utilized features extracted from Magnetic Resonance Imaging (MRI) scans of patients' brains or data pertinent to their gait
~\cite{xia_yao_ye_cheng_2020}~\cite{9207380}~\cite{salmanpour_mojtaba_shamsaei_ghasem_hajianfar_hamid_soltanian_zadeh_arman_rahmim_2022}
~\cite{oscar_pena_nogales_2019}. The objectives of these studies encompass categorizing PD subtypes~\cite{seyed-mohammad_fereshtehnejad_yashar_zeighami_dagher_postuma_2017}
~\cite{salmanpour_mojtaba_shamsaei_ghasem_hajianfar_hamid_soltanian_zadeh_arman_rahmim_2022}, identifying biomarkers of progression~\cite{oscar_pena_nogales_2019}
~\cite{burciu_ofori_archer_wu_pasternak_mcfarland_okun_vaillancourt_2017}, or predicting future PD severity~\cite{xia_yao_ye_cheng_2020}
~\cite{abbasi_seyed_mohammad_fereshtehnejad_yashar_zeighami_kevin_michel_herve_larcher_postuma_dagher_2020}, the rate of disease progression~\cite{shu2021predicting}
~\cite{wang_wu_brown_zhang_liu_han_zuo_cheng_feng_2022}, or specific conditions such as levodopa-induced dyskinesia~\cite{luo_chen_gui_2023}.

Regardless of the specific research goal — predicting the severity, trajectory, or progression rate of PD — MRI scans are a type of data that is frequently used as they provide 
superior visualization of the brain's soft tissues. Moreover, their non-invasive nature and widespread availability render them particularly useful~\cite{a_jon_stoessl_wr_wayne_martin_mckeown_sossi_2011}. 
Consequently, features extracted from MRI scans have been instrumental in numerous PD progression studies, albeit the types of MRI scans employed may vary. Diffusion MRI scans, 
for instance, were utilized by Burciu et al. to ascertain whether free water in the posterior substantia nigra constitutes a biomarker for PD progression~\cite{burciu_ofori_archer_wu_pasternak_mcfarland_okun_vaillancourt_2017}. 
Similarly, aiming to identify biomarkers capable of predicting PD progression, Oscar et al. and Abbasi et al. employed diffusion MRI and Diffusion Tensor Imaging (DTI)
respectively in their analysis~\cite{oscar_pena_nogales_2019}~\cite{abbasi_seyed_mohammad_fereshtehnejad_yashar_zeighami_kevin_michel_herve_larcher_postuma_dagher_2020}. 
Hou et al. utilized functional MRI (fMRI), which measures brain activity through fluctuations in blood flow, to predict PD severity~\cite{hou_luo_yang_ou_song_wei_cao_zhao_wu_shang_et_al._2016}, and multiple studies 
have utilized T1-weighted MRI scans to determine PD progression~\cite{salmanpour_mojtaba_shamsaei_ghasem_hajianfar_hamid_soltanian_zadeh_arman_rahmim_2022}~\cite{shu2021predicting}
~\cite{wang_wu_brown_zhang_liu_han_zuo_cheng_feng_2022}. The conventional structural T1-weighted MRI scans, owing to their simpler processing requirements, are less costly and more accessible than some 
of the other types of MRIs. Additionally, existing software algorithms for image segmentation further enhance the utility of T1 MRIs for studies predicting PD progression~\cite{shu2021predicting}. 
While MRI scans may provide insights into changes in the brain structure associated with the progression of PD, establishing precise metrics for evaluating disease progression is 
equally crucial. Several clinical assessment tools may be applied toward this purpose, including the Hoehn and Yahr (H\&Y) scale~\cite{hoehn_yahr_1998}, the Movement Disorder 
Society's revised version of the Unified Parkinson's Disease Rating Scale (MDS-UPDRS)~\cite{goetz_tilley_shaftman_stebbins_fahn_martinez-martin_poewe_sampaio_stern_dodel_et_al._2008}, 
and the Montreal Cognitive Assessment (MoCA)~\cite{nasreddine_phillips_valérie_bedirian_charbonneau_whitehead_collin_cummings_chertkow_2005}, each offering insights into the patient's state 
at the time of evaluation.

Exploring research that leverages T1-weighted MRI data, we highlight three studies aimed at predicting PD progression. Each employed unique metrics, cohort sizes, and ML techniques. 
All three studies utilized clinical and imaging data from the Parkinson's Progressive Marker Initiative (PPMI) dataset to define their cohorts. Salmanpour et al. adopted a two-pronged 
approach: initially, they performed longitudinal clustering to identify distinct PD progression patterns over four years, and they subsequently predicted 
these identified subtypes (mild, intermediate, or severe) using supervised ML techniques~\cite{salmanpour_mojtaba_shamsaei_ghasem_hajianfar_hamid_soltanian_zadeh_arman_rahmim_2022}. 
T1-weighted MRI scans for the 143 subjects in their study were segmented and radiomic features were extracted. A selection of these features, along with some clinical features, were used to 
train models for both the clustering and prediction tasks. Through clustering, 50 subjects were classified as mild, 38 as intermediate, and 55 as severe. Their trained ML classification models 
were able to predict the progression pattern with a 76.1\% to 79.2\% prediction accuracy. Employing a methodology akin to that of Salmanpour et al., Wang et al. utilized clustering to identify 
biotypes, followed by ML classification techniques to predict these biotypes~\cite{wang_cheng_rolls_dai_gong_du_zhang_wang_liu_wang_et_al_2020}. Their study included 314 subjects with PD and 
145 healthy controls for comparison and evaluation. They preprocessed T1-weighted images to derive volume-based deformation-based morphometry data and selected a subset of neuroanatomic features 
that correlated highly with PD clinical scores as part of their feature selection process. The 314 PD patients were clustered into two distinct groups: biotype 1 and biotype 2, comprising 144 
and 200 patients respectively. Employing ML classifiers, they predicted the biotypes with an accuracy of 84.1\% (progression analysis of the biotypes over a five-year span revealed that 
patients classified under biotype 1 had a significantly higher progression rate, based on their MDS-UPDRS scores, than those under biotype 2). Taking an alternative approach, Shu et al. measured 
the H\&Y score at baseline and again after three years to directly quantify the rate of progression and determine a binary classification of each subject in their study as either `stable' or 
`progression'~\cite{shu2021predicting}. After this classification, they constructed a cohort of 144 subjects equally split between the two categories. Like Salmanpour et al., Shu et al. recognized 
the potential of radiomics and trained supervised ML classifiers on a selection of radiomic features extracted from segmented T1-weighted MRI scans. Aiming to predict PD progression, they developed two models: 
one included only the radiomic features, which achieved an accuracy of 79.5\%, and the other combined radiomic features with clinical features, achieving an accuracy of 83.6\%. 

Shu et al. took an intriguing approach by employing the H\&Y score, commonly used in clinical assessments, as the metric for PD progression, and by training ML models on radiomics features to 
predict disease progression. The performance of their models show promise, which in turn prompts the need for further analysis and raises questions about the study's reproducibility and replicability. 
Arafe et al. attempted to reproduce the results of Shu et al. to answer those questions but were unsuccessful~\cite{Arafe2023.05.05.539590}. Building on these studies, we redefine the problem 
by reorganizing the cohort structures and employing diverse feature selection techniques. Our goal is to rigorously evaluate whether radiomic features extracted from structural MRI scans can 
predict the H\&Y score of PD patients.