Parkinson's disease (PD) involves the progressive degeneration of brain neurons, significantly reducing dopamine and affecting motor control~\cite{fahn_2003}. 
This leads to symptoms like tremors and stiffness, and extends beyond motor functions, underscoring the disorder's multifaceted nature~\cite{xia_yao_ye_cheng_2020}. 
Given the variability in how PD manifests and progresses among individuals, accurately assessing the disease's trajectory and rate of progression, especially in its 
early stages, presents significant challenges~\cite{saeed_compagnone_aviv_strafella_black_lang_masellis_2017}~\cite{seyed-mohammad_fereshtehnejad_yashar_zeighami_dagher_postuma_2017}. 
In response to these challenges, certain studies have employed machine learning (ML) techniques to analyze various types of data beyond traditional demographic and clinical data. 
For instance, some of these studies have utilized features extracted from Magnetic Resonance Imaging (MRI) scans of patients' brains or data pertinent to their gait
~\cite{xia_yao_ye_cheng_2020}~\cite{9207380}~\cite{salmanpour_mojtaba_shamsaei_ghasem_hajianfar_hamid_soltanian_zadeh_arman_rahmim_2022}
~\cite{oscar_pena_nogales_2019}. The objectives of these studies encompass categorizing PD subtypes~\cite{seyed-mohammad_fereshtehnejad_yashar_zeighami_dagher_postuma_2017}
~\cite{salmanpour_mojtaba_shamsaei_ghasem_hajianfar_hamid_soltanian_zadeh_arman_rahmim_2022}, identifying biomarkers of progression~\cite{oscar_pena_nogales_2019}
~\cite{burciu_ofori_archer_wu_pasternak_mcfarland_okun_vaillancourt_2017}, or predicting future PD severity~\cite{xia_yao_ye_cheng_2020}
~\cite{abbasi_seyed_mohammad_fereshtehnejad_yashar_zeighami_kevin_michel_herve_larcher_postuma_dagher_2020}, the rate of disease progression~\cite{shu2021predicting}
~\cite{wang_wu_brown_zhang_liu_han_zuo_cheng_feng_2022}, or specific conditions such as levodopa-induced dyskinesia~\cite{luo_chen_gui_2023}.

Regardless of the specific research goal — predicting the severity, trajectory, or progression rate of PD — MRI scans are a type of data that is frequently used as they provide 
superior visualization of the brain's soft tissues. Moreover, their non-invasive nature and widespread availability render them particularly useful~\cite{a_jon_stoessl_wr_wayne_martin_mckeown_sossi_2011}. 
Consequently, features extracted from MRI scans have been instrumental in numerous PD progression studies, albeit the types of MRI scans employed may vary. Diffusion MRI scans, 
for instance, were utilized by Burciu et al. to ascertain whether free water in the posterior substantia nigra constitutes a biomarker for PD progression~\cite{burciu_ofori_archer_wu_pasternak_mcfarland_okun_vaillancourt_2017}. 
Similarly, aiming to identify biomarkers capable of predicting PD progression, Oscar et al. and Abbasi et al. employed diffusion MRI and Diffusion Tensor Imaging (DTI)
respectively in their analysis~\cite{oscar_pena_nogales_2019}~\cite{abbasi_seyed_mohammad_fereshtehnejad_yashar_zeighami_kevin_michel_herve_larcher_postuma_dagher_2020}. 
Hou et al. utilized functional MRI (fMRI), which measures brain activity through fluctuations in blood flow, to predict PD severity~\cite{hou_luo_yang_ou_song_wei_cao_zhao_wu_shang_et_al._2016}, and multiple studies 
have utilized T1-weighted MRI scans to determine PD progression~\cite{salmanpour_mojtaba_shamsaei_ghasem_hajianfar_hamid_soltanian_zadeh_arman_rahmim_2022}~\cite{shu2021predicting}
~\cite{wang_wu_brown_zhang_liu_han_zuo_cheng_feng_2022}. The conventional structural T1-weighted MRI scans, owing to their simpler processing requirements, are less costly and more accessible than some 
of the other types of MRIs. Additionally, existing software algorithms for image segmentation further enhance the utility of T1 MRIs for studies predicting PD progression~\cite{shu2021predicting}. 
While MRI scans may provide insights into changes in the brain structure associated with the progression of PD, establishing precise metrics for evaluating disease progression is 
equally crucial. Several clinical assessment tools are available for this purpose, including the Hoehn and Yahr (H\&Y) scale~\cite{hoehn_yahr_1998}, the Movement Disorder 
Society's revised version of the Unified Parkinson's Disease Rating Scale (MDS-UPDRS)~\cite{goetz_tilley_shaftman_stebbins_fahn_martinez-martin_poewe_sampaio_stern_dodel_et_al._2008}, 
and the Montreal Cognitive Assessment (MoCA)~\cite{nasreddine_phillips_valérie_bedirian_charbonneau_whitehead_collin_cummings_chertkow_2005}, each offering insights into the patient's state 
at the time of evaluation.

% 3. 

% - who tried to predict progression using any metric with T1s

% - for the ppl who tried to predict progression,
%  - what metric?
%  - which dataset?
%  - how many subjects and what was the division?
%  - what types of features were they looking at?
%  - what kind of results did they get? 

Delving into the realm of research leveraging T1-weighted MRI scans for predicting PD progression, we spotlight three intriguing studies. Each embarks on a unique path, utilizing differing 
PD progression metrics, cohort sizes, and ML techniques. Salmanpour et al. used a two-pronged approach: longitudinal clustering to identify distinct PD progression patterns over four years, 
followed by the prediction of the identified subtypes (mild, intermediate, or severe) utilizing supervised ML techniques on early data~\cite{salmanpour_mojtaba_shamsaei_ghasem_hajianfar_hamid_soltanian_zadeh_arman_rahmim_2022}. 
Subject data for the 143 subjects in their study was obtained from the Parkinson's Progressive Marker Initiative (PPMI) dataset. T1-weighted MRI scans were segmented and the radiomic features were 
extracted and used in combination with clinical features related to both motor and non-motor functions for both the clustering and prediction tasks. Patient data from years 0, 1, 2, and 4 was used for 
the clustering process whereas a new dataset from only years 0 or 1 were used for the prediction task.  


Since some 
information of some patients were missing in some years, 
for longitudinal clustering task, we arrived at dataset of 143 
patients (years 0, 1, 2 and 4). For the prediction task based 
on year 0 and/or 1, we created 3 datasets:


% NOTE: I'm still putting together the rest of the above paragraph but it's a work in progress so I'm not including it here. I should note that I've included the three studies below
% that I've been working on including in this paragraph but I'm not convinced that they're my best option so I'm looking into other studies before I commit. Any feedback here is welcome.


% % studies

% ~\cite{wang_wu_brown_zhang_liu_han_zuo_cheng_feng_2022}
% -progression
% -uses UPDRS
% -uses MRI - T1
% -Model

% ~\cite{salmanpour_mojtaba_shamsaei_ghasem_hajianfar_hamid_soltanian_zadeh_arman_rahmim_2022}
% - T1 MRI
% - longitudal clustering to identify progression
% - subtyping and progression prediction

% ~\cite{shu2021predicting}
% - T1 MRI
% - H&Y score metric
% - progression prediction


% 4. 
% - we consider the interesting nature of the Shu et al study and discuss our concerns with replication and reproduction
% - we ask whether this is really possible to achieve
 
